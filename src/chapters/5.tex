\chapter{Quantum Key Distribution}

Why is Shor's algorithm so important? Most widely used \tbf{public-key cryptography} algorithms rely on the difficulty of one of the three following mathematical problems:

\begin{itemize}
	\item the Integer Factorization problem
	\item the Discrete Logarithm problem
	\item the Elliptic-Curve Discrete Logarithm problem
\end{itemize}

In particular, all of the problems are currently \tit{vary hard} to solve efficiently, and basically any modern cryptosystem is based on the fact that we don't know any fast-enough algorithm that is able to solve them. However, all of these could be easily solved on a sufficiently powerful quantum computer that runs Shor's algorithm, as we already seen. \todo{explain details}

Nevertheless, at the time of writing quantum computers lack the processing power to break widely used cyrptographic algorithm, so we are still safe from a world without crypto. However, because of the length of time required for migration to \tbf{quantum-safe cryptograpy}, are already desining new algorithms to prepare for the so called \tbf{Q-Day}, the day when current algorithms will be vulnerable to quantum computing attacks. For example, NIST already initiated a proces to quotes \curlyquotes{solicit, evaluate and standardize one or more quantum-resistand public-key cryptography algorithms}.

In this chapter we will cover \todo{vedi de che parla prima de parla a vuoto}

is a \tbf{Quantum Key Distribution (QKD)} protocol. In particular, we are interested in finding secure protocols that able to allow two parties to share a \tit{private key} quantum-safely.

\section{The No-cloning theorem} \label{no-cloning}

Before explaining any protocol however, we need to talk about a theorem that we did not properly discuss in \cref{quantum teleportation}, namely the \tbf{No-cloning theorem}. In particular, we stated that there is no quantum transformation that copies \underline{any} quantum state. Now that we have the mathematical tools to prove the theorem, we can actually show the real formulation of the theorem. But first, let's try to understand what it means to \curlyquotes{clone} a qubit. Say that we have a qubit set to some unknown superposition of states $\ket \chi$ that we would like to clone; in other words, we need an ancilla qubit $\ket \gamma$ and some operator $U$ (which must be unitary) such that $$U(\ket \chi \otimes \ket \gamma) = \ket \chi \otimes \ket \chi$$ Essentially, the ancilla qubit is used to store the clone of $\ket \chi$. We observe that the initial state of the ancilla can be considered to be some fixed superposition, since the qubit $\ket \chi$ is \tit{unknown} and we want $U$ to work on any possible state. However, the No-cloning theorem states that such a transformation $U$ cannot exist if the choice of the possible state for $\ket \chi$ isn't restricted enough.

\begin{framedthm}{No-cloning theorem}
	Let $\mathcal H$ be a Hilbert space, and let $S \subseteq \mathcal H$ be a set of vectors. If $S$ contains two distinct non-orthogonal vectors, there is no unitary operator $$\func{U}{\mathcal H \otimes \mathcal H}{\mathcal H \otimes \mathcal H}$$ such that there exists a fixed vector $\ket \gamma \in \mathcal H$ such that $$\forall \ket \chi \in S \quad U(\ket \chi \otimes \ket \gamma) = \ket \chi \otimes \ket \chi$$
\end{framedthm}

\begin{proof}
	By way of contradiction, suppose that there exists a unitary transformation $U$ for which there exists a fixed $\ket \gamma \in \mathcal H$ that can be universally used as ancilla qubit to copy any $\ket \chi \in S$. Moreover, let $\ket \psi, \ket \phi \in S$ two distinct vectors such that $\braket{\psi|\phi} \neq 0$ meaning that they are non-orthogonal. Because $U$ is unitary, by \cref{unitary alt def} we know that it must preserve the scalar product, meaning that $$\forall x, y \in \mathcal H \quad \braket{Ux|Uy} = \braket{x|y}$$ In particular, this must hold for $$x = \ket \psi \otimes \ket \gamma$$ $$y = \ket \phi \otimes \ket \gamma$$ however we see that
	\begin{equation*}
		\begin{alignedat}{2}
			     & \braket{U(\ket \psi \otimes \ket \gamma)| U(\ket \phi \otimes \ket \gamma)} = \braket{ \psi \otimes  \gamma|  \phi \otimes \gamma} &                                   \\
			\iff & \braket{ \psi \otimes \psi | \phi \otimes \phi} = \braket{\psi \otimes \gamma| \phi \otimes \gamma}                                &                                   \\
			\iff & \braket{\psi|\phi} \braket{\psi|\phi} = \braket{\psi| \phi} \braket{\gamma|\gamma}                                                 &                                   \\
			\iff & \braket{\psi|\phi} = 1                                                                                                             & \quad (\braket{\phi|\psi} \neq 0) \\
			\iff & \ket \psi = \ket \phi
		\end{alignedat}
	\end{equation*}
	This means that $\ket \psi$ and $\ket \phi$ are actually the same vector, contradicting the fact that we chose them distinct in $S$ $\lightning$.
\end{proof}

The most important details of this formulation are the following:

\begin{itemize}
	\item we require $S$ to have at least two distinct non-orthogonal vectors
	\item the theorem is defined on a \tit{set} of vectors $S$ which may not necessarily be the whole Hilbert space considered
\end{itemize}

The second observation is crucial because depending on the choice of $S$ \curlyquotes{cloning} is \tit{possible}. Indeed, suppose that we restrict our interest in the set of the only two states $\ket 0$ and $\ket 1$, i.e. $S = \{\ket 0, \ket 1\}$ In this scenario, we \tit{can} actually define an operator such that $$\exists \ket \gamma \quad \forall \ket \chi \in \{\ket 0, \ket 1\} \quad U(\ket \chi \otimes \ket \gamma) = \ket \chi \otimes \ket \chi$$ and it's just a CNOT with $y = \ket 0$: $$\mbox{CNOT}(\ket 0 \otimes \ket 0) = \ket 0 \otimes \ket 0$$ $$\mbox{CNOT}(\ket 1 \otimes \ket 1) = \ket 1 \otimes \ket 1$$ This operation is called \tbf{duplication}, and depending on the set $S$ it can be performed without any issue. As a final note, we observe that the No-cloning theorem is usually stated as follows.

\begin{framedcor}{}
	There is no quantum transformation that copies an unknown quantum state.
\end{framedcor}

This is just the particular case in which $S = \mathcal H$, and $\mathcal H$ obviously contains non-orthogonal pair of distinct states.

The hypothesis of the No-cloning theorem regarding the choice of $S$ is used strategically by the BB84 protocol, as we will see in the next section.

\section{Measurements}

Before describing the next step of the protocol, we need to do a step back and talk about \tit{measurement} in greater detail. Each time we discussed \curlyquotes{measuring} we never actually mentioned that when we are \tit{physically} performing a measurement what we are really doing is applying a \curlyquotes{polarizing lens} to the final state. For instance, this operation can be implemented through \href{https://en.wikipedia.org/wiki/Faraday_rotator}{Faraday rotators}, which leverage the \tbf{Faraday effect}:

\centeredsvg[A Faraday polarization rotator.]{0.5}{../assets/faraday}

This is important because it means that when we talk about measurement we need to define the \curlyquotes{angle} at which we perform it. Mathematically speaking, this is equivalent to defining the \tbf{basis of choice} for the measurement --- we always consider orthonormal bases only, so all the possible bases of a space are identical up to a rotation. Indeed, the importance of the choice of the basis can be illustrated through the BB84 protocol itself.

\begin{itemize}
	\item If we have a qubit in the state $\ket \phi = \ket +$, we know that the probability of getting either $\ket 0$ or $\ket 1$ when measured in the Z basis is precisely 50\% for both outcomes because $$\Pr[\mbox{measure$(\ket \phi = \ket 0)$}] = \abs{\braket{0|+}}^2 = \dfrac{1}{2}$$
	\item Now, consider the orthonormal basis formed by the vector $\ket +$ and $\ket - $ --- which is usually called \tbf{X basis}; we observe that this space is just a rotation of 45° clockwise of the Z basis. This space is interesting: when we measure a qubit $\ket \phi $ set to the state $\ket 0$ in the Z basis we get $\ket 0$ with 100\% certainty, but in this space we have that
	      \begin{equation*}
		      \begin{split}
			      \Pr[\mbox{measure($\ket \phi = \ket +)$}] & = \abs{\braket{0|+}}^2                              \\
			                                                & = \abs{\dfrac{1}{\sqrt 2}\bra 0(\ket 0 + \ket 1)}^2 \\
			                                                & = \abs{\dfrac{1}{\sqrt 2}(1 + 0)}^2                 \\
			                                                & = \dfrac{1}{2}                                      \\
		      \end{split}
	      \end{equation*}
	      Indeed, as we would expect $\ket 0$ and $\ket 1$ in the X basis behave exactly as $\ket +$ and $\ket - $ do in the Z one.
\end{itemize}

We will see how the BB84 protocol takes advantage of this fact later in our discussion. The last detail that we want to mention about measurements in non-diagonal bases is that we can actually define unitary operators that allow us to perform measurements in the Z basis no matter the basis of choice, by first applying some particular unitary operator that creates a \curlyquotes{map} between the two bases. For instance, say that we have some vector $$\ket v = \alpha \ket + + \beta \ket -$$ defined in the X basis, and we want to measure this vector in the Z basis. This can be done by first applying the following operator: $$U = \ket 0 \bra + + \ket 1 \bra -$$ In fact, it holds that
\begin{equation*}
	\begin{split}
		U \ket v & = (\ket 0 \bra + + \ket 1 \bra - )(\alpha \ket + + \beta \ket -)                               \\
		         & = \ket 0 \bra + (\alpha \ket + + \beta \ket - ) + \ket 1 \bra - (\alpha \ket + + \beta \ket -) \\
		         & = \alpha \ket 0 + \beta \ket 1
	\end{split}
\end{equation*}
We can actually generalize this idea: if we have two bases $\{f_i\}_{i = 1}^n$ and $\{g_i\}_{i = 1}^n$ an operator that maps each $f_i$ to $g_i$ for all $i \in [n]$ is exactly $$U_{fg} = \sum_{i = 1}^n \ket{g_i}\bra{f_i}$$ Interestingly enough, we already know an operator that acts as a map between the X and the Z bases:
\begin{equation*}
	\begin{split}
		U & = \ket 0 \bra + + \ket 1 \bra -     \\
		  & = \rmat{1                           \\ 0} \cdot \sbk{\dfrac{1}{\sqrt 2} (\ket 0 + \ket 1)}^\dag + \rmat{0 \\ 1} \cdot \sbk{\dfrac{1}{\sqrt 2} (\ket 0 - \ket 1)}^\dag \\
		  & = \dfrac{1}{\sqrt 2} \rmat{1        \\ 0 } \rmat{1 & 1} + \dfrac{1}{\sqrt 2} \rmat{0 \\ 1} \rmat{1 & -1} \\
		  & = \dfrac{1}{\sqrt 2} \rmat{1    & 1 \\ 0 & 0} + \dfrac{1}{\sqrt 2} \rmat{0 & 0 \\ 1 & -1} \\
		  & = \dfrac{1}{\sqrt 2} \rmat{1    & 1 \\ 1 & -1} \\
		  & = H
	\end{split}
\end{equation*}
This should not come as a suprise, since we already knew that $$H \ket 0 = \ket + \quad \quad H \ket 1 = \ket -$$

\section{The BB84 protocol}

The protocol that we will present is usually called \tbf{BB84}, and it was published in 1984 by \textcite{bb84}. The protocol works as follows: suppose that our two usual protagonists Alice and Bob need to share a key quantum-safely. At the beginning, Alice will choose two random binary strings $(4 + \delta)n$-long --- for some $\delta > 0$ sufficiently large --- which are usually referred to as $a$ and $b$. Then, depending on the bits of these two strings she will construct a superposition of states based on the following table:

\begin{center}
	\begin{tabular}{cc|c}
		\hline
		$a_i$ & $b_i$ & Chosen state \\
		\hline\hline
		0     & 0     & $\ket 0$     \\
		\hline
		1     & 0     & $\ket 1$     \\
		\hline
		0     & 1     & $\ket +$     \\
		\hline
		1     & 1     & $\ket -$     \\
		\hline
	\end{tabular}
\end{center}

For example, if her two random strings are $$a = 01101$$ $$b = 00110$$ then she will construct the following superposition $$\ket \psi = \ket 0 \otimes \ket 1 \otimes \ket - \otimes \ket + \otimes \ket 1$$ Now, we will assume that Alice and Bob share two communication channels:

\begin{itemize}
	\item one quantum channel, which can be eavesdropped by a third party Eve that wants to know their shared key
	\item one public classical \tit{authenticated} channel --- in particular Alice and Bob must be sure about the identity of the other
\end{itemize}

Now that Alice has constructed her superposition $\ket \psi$, she will send it to Bob which will receive $\mathcal E(\ket \psi)$, where $\mathcal E$ describes the quantum operation due to the combined effect of the channel's noise and Eve's actions --- we will describe what Eve can do later in our discussion.

Bob will then proceed to generate a random string of $(4 + \delta)n$ bits as well, which we will call $b'$, and based on this very string he will measure the received quantum superposition --- in particular, if $b_i' = 0$ he will use the Z base, and if $b_i' = 1$ he will employ the X one. In our example, assuming that Bob generated the random string $$b' = 10111$$ he will measure something like $$\mbox{measure}(\ket \psi) = \ket - \otimes \ket 1 \otimes \ket - \otimes \ket + \otimes \ket + $$ We observe that the first and the last states randomly collapsed to $\ket -$ and $\ket +$ respectively because they were measured in the wrong basis --- indeed, the original string was $b = 00110$ and $b \oplus b' = 10001$. Bob then tries to reconstruct the original string $a$ based on its $b'$, thus getting the string $$a' = 11100$$ When this process is complete, Bob publicly announces that he measured the superposition he received in the authenticated channel to Alice.

After Alice has heard that Bob has measured the state, she can proceed to send $b$ itself \tit{as it is} over the authenticated public channel to Bob --- we will see why she is sure that she can perform this operation safely. Through $b$ Bob can then discard all the bits of $a'$ that were generated through wrong bits of $b'$ --- we observe that their remaining bits satisfy $a_i = a'_i$ for all kept indices $i$. This step is usually called \tbf{basis reconciliation}, and the expected number of bits kept by Bob after this step is $$\Exp[\mbox{kept bits}] = \sum_{i = 1}^{(4 + \delta)n}{1 \cdot \Pr[b_i = b_i']} = \sum_{i = 1}^{(4 + \delta)n}{\dfrac{1}{2}} = \dfrac{1}{2} \cdot (4 + \delta)n = \rbk{2 + \dfrac{\delta}{2}}n$$ At this point, Alice and Bob will agree on $2n$ bits to keep from the reconciled string (which can be done again through the public medium) --- $\delta$ can be chosen sufficiently large so that there at least $2n$ bits to choose with exponentially high probability.

At this point, the protocol is basically finished, and the last step involves \tbf{error correction} which has to take into account both the possible noise of the quantum medium, and the potential actions of Eve. Thus, as final step Alice and Bob agree on a split of their $2n$ bits into 2 sets of $n$ bits (choosen UAR), such that half of them will be used as the \tbf{shared key}, and the other half are used as \tit{check bits}. In particular, Alice and Bob publicly share the check bits, such that

\begin{itemize}
	\item if more than $t$ bits disagree, they abort and re-try the protocol from the start
	\item if less that $t$ bits disagree, the error rate is estimated and used to apply \tit{error correction} algorithms to the key bits --- we will not cover the details of the error correction procedures since they are not part of the key distribution protocol itself
\end{itemize}

\subsection{Eve's attacks}

Now, it's time to discuss the Eve's role in the protocol. Suppose that Eve has access to the quantum channel Alice and Bob are using, and wants to understand the key they are trying to share. At the beginning of the protocol Alice sends her $\ket \psi$ which encodes the states she generated through $a$ and $b$, so what happens if Eve can see this qubit? Due to the No-cloning theorem, she cannot clone $\ket \psi$ because in our case we have that $$S = \{\ket 0, \ket 1, \ket +, \ket - \}$$ and half of the possible pairs are actually non-orthogonal. However, she could theoretically do is intercept $\ket \psi$, measure it herself, and send the qubit back to Bob acting as a \tit{man-in-the-middle} --- and Bob would have no way to know this actually happened. Nevertheless, nor Bob neither Eve know the sequence of bases Alice chose originally, so the best thing Eve can do is try randomly (exactly as Bob does in the protocol anyway). Hence, fix a state $\ket \psi_i$ of the states that constitute $\ket \psi$; the following can happen:

\begin{itemize}
	\item Eve chooses the correct basis for $\ket \psi_i$: this means that she does not introduce any disturbance to the state Bob will receive, and Bob still has 50\% chance of recovering Alice's original qubit
	\item Eve chooses the wrong basis for $\ket \psi_i$: this means that she introduces some disturbance to the state Bob will receive, in the sense that now Bob has only 25\% chance of recovering Alice's original qubit
\end{itemize}

For instance, suppose that Alice sends $\ket +$; if Eve guesses the correct basis it holds that
\begin{equation*}
	\hspace{-1cm}
	\begin{split}
		\Pr[\mbox{Bob measures $\ket + $}\mid \mbox{Eve chooses $X$}] & =  \Pr[\mbox{Bob measures $\ket +$} \mid \mbox{Bob chooses $X$}, \mbox{Eve sends $\ket + $}] \\
		                                                              & \quad \cdot \Pr[\mbox{Bob chooses $X$}]                                                      \\
		                                                              & \quad \cdot \Pr[\mbox{Eve sends $\ket +$} \mid \mbox{Eve chooses $X$}]                       \\
		                                                              & =  1 \cdot \dfrac{1}{2} \cdot 1                                                              \\
		                                                              & =  \dfrac{1}{2}                                                                              \\
	\end{split}
\end{equation*}
(obviously, all the probabilities are conditioned under the fact that Alice sent $\ket +$). Differently, if Eve guesses the wrong basis we have that
\begin{equation*}
	\hspace{-1cm}
	\begin{split}
		\Pr[\mbox{Bob measures $\ket + $} \mid \mbox{Eve chooses $Z$}] & = \sum_{b \in \B}(\Pr[\mbox{Bob measures $\ket +$} \mid \mbox{Bob chooses $X$, Eve sends $\ket b$}]       \\
		                                                               & \quad \quad \quad \cdot \Pr[\mbox{Bob chooses $X$}]                                                       \\
		                                                               & \quad \quad \quad \cdot \Pr[\mbox{Eve sends $\ket b$} \mid \mbox{Eve chooses $Z$}] )                      \\
		                                                               & = \dfrac{1}{2} \cdot \dfrac{1}{2} \cdot \dfrac{1}{2} + \dfrac{1}{2} \cdot \dfrac{1}{2} \cdot \dfrac{1}{2} \\
		                                                               & = \dfrac{1}{4}                                                                                            \\
	\end{split}
\end{equation*}
Note that 25\% is \tit{not} the chances that Bob has to recover $\ket +$ \tit{in general}, because that is given by
\begin{equation*}
	\begin{split}
		\Pr[\mbox{Bob measures $\ket +$}] & = \sum_{B \in \{X, Z\}}{\rbk{\Pr[\mbox{Bob measures $\ket + $} \mid \mbox{Eve chooses $B$}] \cdot \Pr[\mbox{Eve chooses $B$}]}} \\
		                                  & = \dfrac{1}{2} \cdot \dfrac{1}{2} + \dfrac{1}{4} \cdot \dfrac{1}{2}                                                             \\
		                                  & = \dfrac{3}{8}                                                                                                                  \\
	\end{split}
\end{equation*}
which is still less than $\tfrac{1}{2}$.

We observe that, on average, with big enough bit strings the chance that Eve correctly chooses each basis can be made exponentially low. Moreover, Eve's actions are exactly the reason why in the last step of the algorithm we perform a thresholded check on the error rate: if the error rate is too high (some noise has to be expected) on average it probably means that someone eavesdropped on the channel.

Is this everything Eve can do? The attack we presented has a very high chance of introducing too much disturbance and being detected by the error rate check step, so for Eve to have some chance of not being detected she needs to \tbf{avoid introducing disturbance completely}. The idea is based on the fact that she actually does not need to apply the full No-cloning theorem: suppose that there exists some unitary operator $U$ that on input $\ket \psi \otimes \ket x$ --- for some state $\ket x$ --- it behaves as follows: $$U(\ket \psi \otimes \ket x) = \ket \psi \otimes \ket y$$ In other words, $U$ leaves $\ket \psi$ unchanged and turns $\ket x$ into $\ket y$. For now, consider $\ket \psi$ as 1 single qubit. If such operator exists, Eve could use it in order to try to measure $\ket y$ in a later moment and gain some information about $\ket \psi$, without ever measuring the latter directly.

This idea seems compelling, however for Eve's attack to be effective $U$ must compute as follows: $$U(\ket \psi \otimes \ket x) = \ket \psi \otimes \ket y$$ $$U(\ket \phi \otimes \ket x) = \ket \phi \otimes \ket{y'}$$ where $\ket y$ and $\ket{y'}$ must be \tbf{different}. In this way

\begin{itemize}
	\item $U$ leaves $\ket \psi$ unchanged, and the latter can be sent to Bob without anyone noticing
	\item $U$ behaves differently depending on the input superposition, i.e. $\ket y \neq \ket{y'}$, otherwise she cannot distinguish between $\ket \psi$ and $\ket \phi$ upon measurement --- to be clear, this has nothing to do with \tit{entanglement}, in fact for Eve it is sufficient to look at the second qubit and infer what the first must have been based on the fact that $\ket y$ and $\ket {y'}$ are distinguishable
\end{itemize}

Indeed, if such $U$ exists she can recover what the original qubit Alice sent only looking at $\ket y$ and using the possible values of the ancilla qubit as \curlyquotes{lookup table}. Let's see if she can employ this strategy.

Suppose that Alice sends some qubit which can be either $\ket \psi$ or $\ket \phi$, where $$\ket \psi, \ket \phi \in \{\ket 0, \ket 1, \ket + , \ket -\}$$ Since $U$ must be unitary, by \cref{unitary alt def} we know that $U$ preserves the scalar product, meaning that
\begin{equation*}
	\begin{alignedat}{2}
		     & \braket{U(\ket \psi \otimes \ket x)| U(\ket \phi \otimes \ket x)} = \braket{ \psi \otimes  x|  \phi \otimes x} & \\
		\iff & \braket{ \psi \otimes y | \phi \otimes y'} = \braket{\psi \otimes x| \phi \otimes x}                           & \\
		\iff & \braket{\psi|\phi} \braket{y|y'} = \braket{\psi| \phi} \braket{x|x}                                            & \\
		\iff & \braket{\psi|\phi} \braket{y|y'} = \braket{\psi| \phi}                                                         & \\
	\end{alignedat}
\end{equation*}
We observe that between all the possible choices of pairs of $\ket \psi$ and $\ket \phi$, half of them are such that $\braket{\psi|\phi} \neq 0$, i.e. $\ket \psi$ and $\ket \phi$ are \tit{non-orthogonal}. This means that we can simplify the last equality even more, obtaining that
\begin{equation*}
	\begin{alignedat}{2}
		     & \braket{\psi|\phi} \braket{y|y'} = \braket{\psi| \phi} &                                   \\
		\iff & \braket{y|y'} = 1                                      & \quad (\braket{\psi|\phi} \neq 0) \\
		\iff & \ket{y} = \ket{y'}                                     &                                   \\
	\end{alignedat}
\end{equation*}
Once again, through an argument fairly similar of the one we used for the No-cloning theorem, this proves that such an operator $U$ cannot exist, which means that there is no way for Eve to gain any information at all without introducing disturbance.

% Furthermore, we observe that this condition of non-orthogonality is precisly why the BB84 protocol chooses the Z and the X basis, which are non-orthogonal between each other. In fact, let's see what happens if the choice of Alice was restricted to only vectors $\ket \psi$ and $\ket \phi$ that are orthogonal. For instance, say that $$\ket \psi, \ket \phi \in \{\ket 0, \ket 1\}$$ In this scenario, Eve actually \tit{has chance} to recover some information without introducing disturbance. Indeed, a transformation $U$ such that $$U (\ket 0 \otimes \ket 0) = \ket 0 \otimes \ket 0$$ $$U (\ket 1 \otimes \ket 0) = \ket 1 \otimes \ket 1$$ is literally just a CNOT on $\ket 0$ where Alice's qubit is the control bit. This shows that the non-orthogonality condition guarantees the security of the protocol itself.

To be precise, there are more subtle attacks that do employ \tit{quantum entagnlement}, such as \tbf{optimal intercept–resend tradeoffs}, where Eve applies a unitary operator that \tit{partially entangles} the signal with a probe to trade a small amount of information for a small disturbance. It can be proven that the BB84 protocol is also resistant against these type of attacks, however this is beyond the scope of our discussion.

Lastly, we observe that BB84 requires Alice to wait before hearing back from Bob that he \tit{actually measured} $\ket \psi$. This is to prevent that Alice sends $b$ too early, so that Eve could use the latter to decode $\ket \psi$ into the original string $a$ without any issue. Therefore, we require Alice to wait for Bob's measurement so that there is no way for Eve to recover $a$ after looking at $b$ becasuse the original superposition is definitively destroyed --- unless she can go back in time!
