\chapter{Quantum Key Distribution}

Why is Shor's algorithm so important? Most widely used \tbf{public-key cryptography} algorithms rely on the difficulty of one of the three following mathematical problems:

\begin{itemize}
	\item the Integer Factorization problem
	\item the Discrete Logarithm problem
	\item the Elliptic-Curve Discrete Logarithm problem
\end{itemize}

In particular, all of the problems are currently \tit{vary hard} to solve efficiently, and basically any modern cryptosystem is based on the fact that we don't know any fast-enough algorithm that is able to solve them. However, all of these could be easily solved on a sufficiently powerful quantum computer that runs Shor's algorithm, as we already seen. \todo{explain details}

Nevertheless, at the time of writing quantum computers lack the processing power to break widely used cyrptographic algorithm, so we are still safe from a world without crypto. However, because of the length of time required for migration to \tbf{quantum-safe cryptograpy}, are already desining new algorithms to prepare for the so called \tbf{Q-Day}, the day when current algorithms will be vulnerable to quantum computing attacks. For example, NIST already initiated a proces to quotes \curlyquotes{solicit, evaluate and standardize one or more quantum-resistand public-key cryptography algorithms}.

In this chapter we will cover \todo{vedi de che parla prima de parla a vuoto}

is a \tbf{Quantum Key Distribution (QKD)} protocol. In particular, we are interested in finding secure protocols that able to allow two parties to share a \tit{private key} quantum-safely.

The protocol that we will present is usually called BB84, and it was published in 1984 by \textcite{bb84}. Suppose that our two usual protagonists Alice and Bob need to share a key quantum-safely. Alice will choose two random binary strings $(4 + \delta)n$-long which are usually referred to as $a$ and $b$. Then, depending on the bits of these two strings she will construct a superposition of states based on the following table:

\begin{center}
	\begin{tabular}{cc|c}
		\hline
		$a_i$ & $b_i$ & Chosen state \\
		\hline\hline
		0     & 0     & $\ket 0$     \\
		\hline
		1     & 0     & $\ket 1$     \\
		\hline
		0     & 1     & $\ket +$     \\
		\hline
		1     & 1     & $\ket - $    \\
		\hline
	\end{tabular}
\end{center}

For example, if her two random strings are $$a = 01101$$ $$b = 00110$$ then she will construct the following superposition $$\ket \psi = \ket 0 + \ket 1 + \ket - + \ket + + \ket 1$$ Now, we will assume that Alice and Bob share a communication channel that can be eavesdropped by a third party Eve that wants to know their shared key. Alice will send $\ket \psi$ to Bob through this medium. Before describing what happens when Eve comes into play, we will describe the complete protocol.

Let's do a step back: each time we talked about \curlyquotes{measuring} we never actually mentioned that when we are \tit{physically} performing a measure what we are really performing is applying a \curlyquotes{polarizing lens} to the final state. For instance, this operation can be implemented through \href{https://en.wikipedia.org/wiki/Faraday_rotator}{Faraday rotators}, which leverage the \tbf{Faraday effect}:

\centeredsvg[A Faraday polarization rotator.]{0.5}{../assets/faraday}

This is important because it means that when we talk about measurement we need to define the \curlyquotes{angle} at which we perform it. Mathematically speaking, this is equivalent to defining the \curlyquotes{base of choice} for the measurement --- since we always consider orthonormal basis only. The importance of the choice of the basis can be illustrated with the BB84 protocol.

\begin{itemize}
	\item If we have a qubit in the state $\ket \phi = \ket +$, we know that the probability of getting either $\ket 0$ or $\ket 1$ when measured in the canonical basis --- formed by the vector $\ket 0$ and $\ket 1$ themselves --- is precisely 50\% for both outcomes because $$\Pr[\mbox{measure$(\ket \phi = \ket 0)$}] = \abs{\braket{0|+}}^2 = \dfrac{1}{2}$$
	\item Now, consider the orthonormal basis formed by the vector $\ket +$ and $\ket - $ --- which we will call the \tbf{diagonal basis}; we observe that this space is just a rotation of $45°$ clockwise. This space is interesting: when we measure a qubit $\ket \phi $ set to the state $\ket 0$ in the canonical base we get $\ket 0$ with 100\% certainty, but in this space we have that
	      \begin{equation*}
		      \begin{split}
			      \Pr[\mbox{measure($\ket \phi = \ket +)$}] & = \abs{\braket{0|+}}^2                              \\
			                                                & = \abs{\dfrac{1}{\sqrt 2}\bra 0(\ket 0 + \ket 1)}^2 \\
			                                                & = \abs{\dfrac{1}{\sqrt 2}(1 + 0)}^2                 \\
			                                                & = \dfrac{1}{2}                                      \\
		      \end{split}
	      \end{equation*}
	      Indeed, as we would expect $\ket 0$ and $\ket 1$ in the diagonal basis behave exactly as $\ket +$ and $\ket - $ do in the canonical one.
\end{itemize}

We will see how the BB84 protocol takes advantage of this fact later in our discussion.

Now that Alice has sent her superposition $\ket \psi$, she will send it to Bob which will receive $\mathcal E(\ket \psi)$, where $\mathcal E$ describes the quantum operation due to the combined effect of the channel and Eve's actions. If Bob correctly receives the information, he publicly announces this fact.

Bob will then proceed to generate a random string of $(4 + \delta)n$ bits as well, which we will call $b'$, and based on this very string he will measure the received quantum superposition --- in particular, if $b_i' = 0$ he will use the canonical base, and if $b_i' = 1$ he will employ the diagonal one. In our example, assuming that Bob generated the random string $$b' = 10111$$ he will measure something like $$\mbox{measure}(\ket \psi) = \ket - + \ket 1 + \ket - + \ket + + \ket + $$ We observe that the first and the last bit randomly collapsed to $\ket -$ and $\ket +$ respectively because they were measured in the wrong basis --- indeed, the original string was $b = 00110$ and $b \oplus b' = 10001$. Bob then tries to reconstruct the original string $a$ based on its $b'$, thus getting the string $$a' = 11100$$ We observe that Bob this string of no use for now, and this is important when we will discuss Eve's role in the protocol.

After Alice has heard that Bob received the state, she can proceed to send $b$ itself as it is over a public channel, and Bob can then discard all the bits of $a'$ that were generated through wrong bits of $b'$ --- we observe that their remaining bits satisfy $a = a'$. This step is usually called \tbf{basis reconciliation}, and the expected number of qubits kept by Bob after this step is $$\Exp[\mbox{kept qubits}] = \sum_{i = 1}^{(4 + \delta)n}{1 \cdot \Pr[b_i = b_i']} = \sum_{i = 1}^{(4 + \delta)n}{\dfrac{1}{2}} = \dfrac{1}{2} \cdot (4 + \delta)n = \rbk{2 + \dfrac{\delta}{2}}n$$ We will assume that both Alice and Bob will just weep $2n$ bits of their result --- $\delta$ can be chosen sufficiently large so that this can be done with exponentially high probability.

% Afterwards, Alice selects $n$ bits

TODO \todo{what happens if eve looks}

