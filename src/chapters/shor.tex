\chapter{Shor's algorithm}

TODO \todo{introduction importance alg + connection w qpe}

TODO \todo{talk about RSA and wtf is shor's alg}

\section{Order-finding problem}

The last ingredient that we need for Shor's algorithm is the \tbf{order-finding} algorithm. We recall the following property of number theory.

\begin{framedprop}{Euclidean division}
    Given some $p \in \N_{> 0}$, for any $n \in \N$ there exists unique integers $q, r \in \Z$ such that $0 \le r < p$ and $$n = p \cdot q + r$$
\end{framedprop}

This is the usual division with remainder, which defines modular arithmetic, for instance $$31 = 4 \cdot 7 + 3$$ Indeed, with equivalence classes we would write that $$31 \equiv 3 \bmod 7$$

\begin{frameddefn}{Order of a number}
    Given a number $N \in \N$, for any $x \in \Z$ the \tbf{order} of $x$ modulo $N$ is the least integer $r$ such that $$x^r \equiv 1 \bmod N$$
\end{frameddefn}

For instance, the order of 4 modulo 7 is 3, because $$4^3 = 64 = 9 \cdot 7 + 1 \implies 4^3 \equiv 1 \bmod 7$$ We observe that throughout our discussion $N$ is \tit{not necessarily} a power of 2 as we used to assume in previous sections, however the symbol $N$ is conventionally used in this context.

Given a number $N$, and a number $x < N$ that is coprime with $N$, can we compute its order modulo $N$? This is the so called \tbf{order-finding problem}, and currently there is no classical algorithm able to solve it in polynomial time. Let's see what quantum computation can achieve.

Consider the following unitary operator $U_x$ that computes as follows: $$\forall y \in \{0, 1\}^k \quad U_x \ket y = \soe{ll}{\ket{xy \bmod N} & y < N \\ \ket y & y \ge N}$$ To derive the complete operator $U_x$ we just follow \cref{U constr}:
\begin{equation*}
    \begin{split}
        U_x & = \sum_{y \in \{0, 1\}^k}{U_x \ket y \bra y} \\ 
            & = \sum_{y < N}{U_x \ket y \bra y} + \sum_{y \ge N}{U_x \ket y \bra y} \\ 
            & = \sum_{y < N}{U_x \ket y \bra y} + \sum_{y \ge N}{\ket y  \bra y} \\ 
    \end{split}
\end{equation*}
We will not replace $U_x$ inside the first sum for now in order to prove that $U_x$ is unitary. But first, we need to present a result in number theory.

\begin{framedthm}[label={number theory}]{}
    If $x$ and $N$ are coprime, it holds that $$\forall y, z \in \Z \quad xy \equiv xz \bmod N \iff y \equiv z \bmod N$$
\end{framedthm}

\begin{proof}
    Since the multiplication modulo $N$ is well-defined, the converse implication is trivial. Moreover, we observe that the direct implication reduces to proving that $x$ is invertible modulo $N$, because if $x^{-1}$ exists modulo $N$ then $$x^{-1} \cdot xy \equiv x^{-1} \cdot xz \bmod N \implies y \equiv z \bmod N$$ Therefore, let $x$ be a number coprime with $N$, i.e. $\gcd(x, N) = 1$. Then, by Bézout's identity it follows that $$\exists a, b \in \Z \quad xa + Nb = 1$$ but this immediately implies that
    \begin{equation*}
        \begin{split}
            & xa + Nb = 1 \\ 
            \iff & xa - 1 = -N b \\ 
            \iff & xa - 1 \equiv 0 \bmod N \\ 
            \iff & xa \equiv 1 \bmod N \\ 
        \end{split}
    \end{equation*}
    which indeed proves that $a$ is the inverse of $x$ modulo $N$.
\end{proof}

\begin{framedprop}{}
    The operator $U_x$ is unitary, and in particular $$U_x^\dag = \sum_{y < N}{\ket y (U_x \ket y)^\dag} + \sum_{y \ge N}{\ket y \bra y}$$
\end{framedprop}

\begin{proof}
    It is easy to prove the correctness of $U_x^\dag$, so we are going to prove that $U_x$ is unitary directly. \todo{this proof is to be rewritten}
    \begin{equation*}
        \begin{split}
            U_x^\dag U_x & = \rbk{\sum_{y < N}{\ket y (U_x \ket y)^\dag} + \sum_{y \ge N}{\ket y \bra y}}\rbk{\sum_{z < N}{U_x \ket z \bra z} + \sum_{z \ge N}{\ket z  \bra z}} \\ 
                         & = \rbk{\sum_{y < N}{\ket y \bra{U_x y}} + \sum_{y \ge N}{\ket y \bra y}} \rbk{\sum_{z < N}{\ket{U_x z} \bra z} + \sum_{z \ge N}{\ket z \bra z}} \\ 
                         & = \sum_{y, z < N}{\ket y \braket{U_x y | U_x z} \bra z} + \sum_{\substack{y < N \\ z \ge N}}{\ket y \braket{U_x y| z} \bra z} + \sum_{\substack{y \ge N \\ z < N}}{\ket y \braket{y |U_xz} \bra z} + \sum_{y, z \ge N}{\ket y \braket{y|z} \bra z} \\ 
        \end{split}
    \end{equation*}
    Now note that if $z \ge N$ and $y < N$, by definition of $U_x$ we have that $U_x \ket y = \ket{xy \bmod N}$, hence it will be a basis state among $\ket 0, \ldots, \ket{N - 1}$. Therefore, if $z > N$ we are guaranteed that $\ket z$ and $\ket{U_x y}$ are orthogonal, i.e. $\braket{U_xy|z} = 0$ --- we recall that $N$ is \tit{not} the size of the space in this context, is just a composite number, indeed the size of the space we are considering is $2^k$. Therefore, we have that $$ \sum_{\substack{y < N \\ z \ge N}}{\ket y \braket{U_x y| z} \bra z} = \sum_{\substack{y \ge N \\ z < N}}{\ket y \braket{y |U_xz} \bra z} = 0$$ so we conclude that
    \begin{equation*}
        \begin{alignedat}{2}
            U_x^\dag U_x & = \sum_{y, z  < N}{\ket y \braket{U_x y | U_x z} \bra z} + \sum_{\substack{y < N \\ z \ge N}}{\ket y \braket{U_x y| z} \bra z} + \sum_{\substack{y \ge N \\ z < N}}{\ket y \braket{y |U_xz} \bra z} + \sum_{y, z \ge N}{\ket y \braket{y|z} \bra z} & \\ 
                         & = \sum_{y, z < N}{\ket y \braket{U_x y | U_x z} \bra z} + \sum_{y, z \ge N}{\ket y \braket{y|z} \bra z} & \\ 
                         & = \sum_{\substack{y, z < N : \\ y \equiv z \bmod N}}{\ket y \braket{U_xy |U_xz} \bra z} + \sum_{\substack{y, z < N : \\ y \not\equiv z \bmod N}}{\ket y \braket{U_xy |U_xz} \bra z} + \sum_{y , z \ge N}{\ket y \braket{y|z} \bra z} & \\ 
        \end{alignedat}
    \end{equation*}
    To progress, we observe that when $y, z < N$ we get that $$\braket{U_x y | U_x z} = \braket{xy \bmod N | xz \bmod N} = \soe{ll}{1 & xy \equiv xz \bmod N \\ 0 & \mbox{otherwise}}$$ and by \cref{number theory} we know that $$xy \equiv xz \bmod N \iff y \equiv z \bmod N$$ thus getting
    \begin{equation*}
        \begin{alignedat}{2}
            U_x^\dag U_x & = \sum_{\substack{y, z < N : \\ y \equiv z \bmod N}}{\ket y \braket{U_xy |U_xz} \bra z} + \sum_{\substack{y, z < N : \\ y \not\equiv z \bmod N}}{\ket y \braket{U_xy |U_xz} \bra z} + \sum_{y , z \ge N}{\ket y \braket{y|z} \bra z} & \\ 
                         & = \sum_{\substack{y, z < N : \\ y \equiv z \bmod N}}{\ket y  \bra z} + \sum_{y , z \ge N}{\ket y \braket{y|z} \bra z} & \\ 
                         & = \sum_{\substack{y, z < N : \\ y = z}}{\ket y  \bra z} + \sum_{y , z \ge N}{\ket y \delta_{yz} \bra z} \\ 
                         & = \sum_{\substack{y < N}}{\ket y  \bra y} + \sum_{y \ge  N}{\ket y  \bra y} &  \\
                         & = \sum_{y}{\ket y \bra y} & \\ 
                         & = I & \\ 
        \end{alignedat}
    \end{equation*}

    Now, we need to prove the opposite product \todo{da fare}
\end{proof}

We did not provide any reason to why we defined the operator $U_x$ as such, but before giving a geometrical intuition consider the following proposition.

\begin{framedthm}{}
    Given $N \in \N$, and $x \in [0, N - 1]$, if $r$ is the order of $x$ modulo $N$, it holds that $$\forall s \in [0, r - 1] \quad \ket{u_s} = \dfrac{1}{\sqrt r} \sum_{k = 0}^{r - 1}{e^{-2 \pi i sk/r} \ket{x^k \bmod N}}$$ is an eigenvector of $U_x$.
\end{framedthm}

\begin{proof}
    Fix $s \in [0, r - 1]$; to prove that $\ket{u_s}$ is an eigenvector of $U_x$ we need to show that there exists some phase $\varphi_s$ such that $$U_x \ket{u_s} = e^{2 \pi i \varphi_s} \ket{u_s}$$ because of \cref{spectral thm}. Hence, we get that
    \begin{equation*}
        \hspace{-1cm}
        \begin{alignedat}{2}
            U_x \ket{u_s} & = U_x \rbk{\dfrac{1}{\sqrt r} \sum_{k = 0}^{r - 1}{e^{-2 \pi i sk/r} \ket{x^k \bmod N}}} & \\ 
                          & = \dfrac{1}{\sqrt r} \sum_{k = 0}^{r - 1}{e^{-2 \pi i sk/r} U_x \ket{x^k \bmod N}} & \\ 
                          & = \dfrac{1}{\sqrt r} \sum_{k = 0}^{r - 1}{e^{-2 \pi i sk/r}  \ket{x\rbk{x^k \bmod N} \bmod N}} & \quad \rbk{x^k \bmod N < N} & \\ 
                          & = \dfrac{1}{\sqrt r} \sum_{k = 0}^{r - 1}{e^{-2 \pi i sk/r}  \ket{x^{k + 1} \bmod N \bmod N}} & \\ 
                          & = \dfrac{1}{\sqrt r} \sum_{k = 0}^{r - 1}{e^{-2 \pi i sk/r}  \ket{x^{k + 1} \bmod N}} &  \\ 
                          & = \dfrac{1}{\sqrt r} \cdot \dfrac{e^{2 \pi i s/r}}{e^{2 \pi i s/r}} \cdot \sum_{k = 0}^{r - 1}{e^{-2 \pi i sk/r} \ket{x^{k + 1} \bmod N}} & \\ 
                          & = \dfrac{1}{\sqrt r}  e^{2 \pi i s/r} \cdot \sum_{k = 0}^{r - 1}{e^{-2 \pi i s(k + 1)/r}  \ket{x^{k + 1} \bmod N}} & \\ 
                          & = \dfrac{1}{\sqrt r} \cdot e^{2 \pi i s/r} \rbk{\sum_{k = 0}^{r - 2}{e^{-2 \pi i s(k + 1)/r}  \ket{x^{k + 1} \bmod N}} + e^{ - 2 \pi i s r /r} \ket{x^r \bmod N}} & \\ 
                          & = \dfrac{1}{\sqrt r}  e^{2 \pi i s/r} \rbk{\sum_{k = 0}^{r - 2}{e^{-2 \pi i s(k + 1)/r}  \ket{x^{k + 1} \bmod N}} + e^{ - 2 \pi i s} \ket{1}} & \quad (x^r \equiv 1 \bmod N) \\ 
                          & = \dfrac{1}{\sqrt r} e^{2 \pi i s/r} \rbk{\sum_{m = 1}^{r - 1}{e^{-2 \pi i s m /r}  \ket{x^m \bmod N}} +  \ket{1}} & \\ 
                          & = \dfrac{1}{\sqrt r}  e^{2 \pi i s/r} \rbk{\sum_{m = 0}^{r - 1}{e^{-2 \pi i s m /r}  \ket{x^m \bmod N}}} & \quad (\ket 1 = \ket{x^0 \bmod N})\\ 
                          & = e^{2 \pi i s/r}  \ket {u_s} \\
        \end{alignedat}
    \end{equation*}
    Therefore, we conclude that $\varphi_s = s/r$.
\end{proof}

TODO \todo{intuizione geometrica}

What happenns if the input of $U_x$ is $\ket{x^0}$? By definition of our problem $x < N$, therefore $$U_x \ket{x^0} = \ket{x \cdot x^0 \bmod N} = \ket{x^1 \bmod N}$$ Indeed in general it's easy to see that $$U_x \ket{x^k} = \ket{x^{k + 1} \bmod N}$$ In other words $U_x$ is cycling through $x$'s powers, which also implies that when $k = r - 1$ we get that $$U_x \ket{x^{r - 1}} = \ket{x^r \bmod N} = \ket{1 \bmod N}$$ since $r$ is the order of $x$ modulo $N$. Moreover, as we already know these are basis states so they are both normalized and orthogonal to each other, thus the powers $x$ form an orthonormal base of the following space $$\mathcal H_r = \mbox{span}\rbk{\ket{x^k \bmod N} \mid k \in [0, r - 1]}$$ which is a restriction of the whole Hilbert space that has exactly $r$ dimensions. Now look again at how the eigenvectors of $U_x$ are defined: $$\forall s \in [0, r - 1] \quad \ket{u_s} = \dfrac{1}{\sqrt r} \sum_{k = 0}^{r - 1}{e^{-2 \pi i sk/r} \ket{x^k \bmod N}}$$ This looks very similar to how we defined $\mbox{QFT} \ket x$: $$\mbox{QFT} \ket x = \dfrac{1}{\sqrt N} \sum_{k = 0}^{N - 1}{e^{2 \pi i x k / N} \ket k}$$ Indeed, it holds that $\ket{u_s}$ is basically $\mbox{QFT} \ket s$ but taken inside $\mathcal H_r$ --- the only difference being the sign of the exponent, ndeed it technically holds that $$\ket{u_s} = \mbox{QFT}_r \ket{- s \bmod r}$$ but as we said the sign of the exponent is just a convention either sign is found in literature, we only need to be consistent with the calculations. \todo{non ho capito che c'entra però}

The most interesting part is that, as proved in the last theorem, for any fixed $s \in [0, N - 1]$ its associated phase $\varphi_s$ is exactly $s/r$, thus if we knew how to prepare the last register of the QPE as $\ket{u_s}$ we could recover its phase, and maybe get closer to know $r$ itself. However, we have two probems with this idea:

\begin{itemize}
    \item there is really no easy or practical way to prepare the second register to some arbitrary state
    \item $\ket{u_s}$ actually depends on $r$ itself, which actually is a very big problem on its own
\end{itemize}

Thankfully, the following property will solve both of these issues at the same time.

\begin{framedprop}{}
    Given $N \in \N$, and $x \in [0, N - 1]$, if $r$ is the order of $x$ modulo $N$ it holds that $$\dfrac{1}{\sqrt r} \sum_{s = 0}^{r - 1} \ket{u_s} = \ket 1$$
\end{framedprop}

\begin{proof}
    Through some algebraic manipulation we see that
    \begin{equation*}
        \begin{split}
            \dfrac{1}{\sqrt r} \sum_{s = 0}^{r - 1}{\ket{u_s}} & = \dfrac{1}{\sqrt r} \sum_{s = 0}^{r - 1}{\rbk{\dfrac{1}{\sqrt r} \sum_{k = 0}^{r - 1}{e^{- 2 \pi i s k / r} \ket {x^k \bmod N}}}} \\ 
                                                               & = \dfrac{1}{r} \sum_{s, k = 0}^{r - 1}{e^{- 2 \pi i sk/ r} \ket{x^k \bmod N}} \\ 
                                                               & = \dfrac{1}{r} \sum_{s = 0}^{r - 1}{e^{- 2 \pi i s \cdot 0 / r} \ket{x^0 \bmod N}} + \dfrac{1}{r} \sum_{s = 0, k = 1}^{r - 1}{e^{- 2 \pi i sk/ r} \ket{x^k \bmod N}} \\ 
                                                               & = \dfrac{1}{r} \sum_{s = 0}^{r - 1}{\ket{1}} + \dfrac{1}{r} \sum_{s = 0, k = 1}^{r - 1}{e^{- 2 \pi i sk/ r} \ket{x^k \bmod N}} \\ 
                                                               & = \dfrac{1}{r} r \ket 1 + \dfrac{1}{r} \sum_{s = 0, k = 1}^{r - 1}{e^{- 2 \pi i sk/ r} \ket{x^k \bmod N}} \\ 
                                                               & = \ket 1 + \dfrac{1}{r}\sum_{k = 1}^{r - 1}{\ket{x^k \bmod N} \sum_{s = 0}^{r - 1}{e^{- 2 \pi i sk/r }}} \\ 
                                                               & = \ket 1 + \dfrac{1}{r}\sum_{k = 1}^{r - 1}{\ket{x^k \bmod N} \sum_{s = 0}^{r - 1}{\rbk{e^{- 2 \pi i k/r}}^s}} \\ 
        \end{split}
    \end{equation*}
    As we did for the proof of \cref{qft unitary}, because of how we split the sums we know that $k \neq 0$ hence $e^{-2 \pi i k/r} \neq 1$ because $k/r$ is not an integer (since $k \in [1, r - 1]$), therefore we get that
    \begin{equation*}
        \begin{split}
            \dfrac{1}{\sqrt r} \sum_{s = 0}^{r - 1}{\ket{u_s}} & = \ket 1 + \dfrac{1}{r}\sum_{k = 1}^{r - 1}{\ket{x^k \bmod N} \dfrac{1 - (e^{-2 \pi i k/r})^r}{1 - e^{- 2 \pi i k/r}}} \\ 
                                                               & = \ket 1 + \dfrac{1}{r}\sum_{k = 1}^{r - 1}{\ket{x^k \bmod N} \dfrac{1 - e^{-2 \pi i k}}{1 - e^{- 2 \pi i k/r}}} \\ 
                                                               & = \ket 1 + \dfrac{1}{r}\sum_{k = 1}^{r - 1}{\ket{x^k \bmod N} \dfrac{1 - 1}{1 - e^{- 2 \pi i k/r}}} \\ 
                                                               & = \ket 1
        \end{split}
    \end{equation*}
\end{proof}

Indeed, this property is incredibly useful because we can provide $\ket 1$ to the lower register of the QPE circuit instead of a single eigenvector, in order to compute the same algorithm but to all the eigenvectors \tbf{simultaneously}. We observe that $\ket 1$ is essentially a superposition of all the eigenvectors of $U_x$, which means that when fed to $\mbox{C-}U_x^{2^k}$ as target qubit we don't really get $\ket 1$ back, indeed
\begin{equation*}
    \begin{split}
        \mbox{C-}U_x^{2^k} (\ket{+}_0 \otimes \ket{1}_1) & = \dfrac{1}{\sqrt 2} \rbk{\ket{0}_0 \otimes \ket{1}_1 + \ket{1}_0 \otimes U_x^{2^k} \ket 1_1} \\ 
                                                         & = \dfrac{1}{\sqrt 2} \rbk{\ket{0}_0 \otimes \ket{1}_1 + \ket{1}_0 \otimes U_x^{2^k} \dfrac{1}{\sqrt r} \sum_{s = 0}^{r - 1}{\ket{u_s}_1}} \\ 
                                                         & = \dfrac{1}{\sqrt 2} \rbk{\ket{0}_0 \otimes \ket{1}_1 + \ket{1}_0 \otimes \dfrac{1}{\sqrt r} \sum_{s = 0}^{r - 1}{U_x^{2^k}\ket{u_s}_1}} \\ 
                                                         & = \dfrac{1}{\sqrt 2} \rbk{\ket{0}_0 \otimes \ket{1}_1 + \ket{1}_0 \otimes \dfrac{1}{\sqrt r} \sum_{s = 0}^{r - 1}{e^{2 \pi i 2^k s/r} \ket{u_s}_1}} \\ 
                                                         & = \dfrac{1}{\sqrt 2} \rbk{\ket{0}_0 \otimes \dfrac{1}{\sqrt r}\sum_{s = 0}^{r - 1}{\ket{u_s}_1} + \ket{1}_0 \otimes \dfrac{1}{\sqrt r} \sum_{s = 0}^{r - 1}{e^{2 \pi i 2^k s/r} \ket{u_s}_1}} \\ 
                                                         & = \dfrac{1}{\sqrt{2r}} \sum_{s = 0}^{r - 1}{\rbk{\ket{0}_0 + e^{2 \pi i 2^k s/r} \ket{1}_0} \otimes \ket{u_s}_1} \\
    \end{split}
\end{equation*}
Nevetheless, notice what happened here: we engangled the control qubit with the target qubit, such that now the target \tit{picked up the phase} $e^{2 \pi i 2^k s/r}$ --- exactly as the standard QPE --- and we are still preserving the superposition of eigenvectors $\ket{u_s}$. This is exactly what we need, however it also means that the result at the end of the circuit is not very straightforward: in standard QPE the resulting state is just the tensor product of the single control qubits because they are independent from each other, but here we are entangling all the control and the target qubits together at each application of $\mbox{C-}U_x^{2^k}$. Hence, we need to compute what happens step by step: \todo{fallo}
\begin{equation*}
    \begin{split}
        & TODO \\ 
        = & \dfrac{1}{\sqrt r} \sum_{s = 0}^{r - 1}{\dfrac{1}{\sqrt{2^t}} \bigotimes_{m = 1}^{t}{\rbk{\ket 0 + e^{2 \pi i 2^{m - 1} s/r} \ket 1}} \otimes \ket{u_s}} \\  
        = & \dfrac{1}{\sqrt r} \sum_{s = 0}^{r - 1}{\dfrac{1}{\sqrt{2^t}} \bigotimes_{m = 1}^{t}{\rbk{\ket 0 + e^{2 \pi i 0.\varphi_{s_m} \ldots \varphi_{s_t}} \ket 1}} \otimes \ket{u_s}} \\ 
        = & \dfrac{1}{\sqrt r} \sum_{s = 0}^{r - 1}{\mbox{QFT}_t \ket{\varphi_s}_{q_0 \cdots q_{t - 1}} \otimes \ket{u_s}} \\ 
        \xrightarrow{\mathrm{QFT}_t^\dag (\ket{q_0 \cdots q_{t - 1}})} & \dfrac{1}{\sqrt r} \sum_{s = 0}^{r - 1}{\ket{\varphi_s} \otimes \ket{u_s}} \\ 
    \end{split}
\end{equation*}
assuming that $\varphi_s = s/r$ can be written through $t$ bits. This entangled quantum state is exactly what we want, because now by measuring the first $t$ registers we get that for each $s \in [0, r - 1]$ they will collapse into $\ket{\varphi_s}$ with probability $\tfrac{1}{r}$, and the last register will collapse into $\ket{u_s}$ which is exactly the eigenvector associated to $e^{2 \pi i \varphi_s}$. This proves that we truly computed the phases of all the possible eigenvectors simultaneously.

We are almost done. We now know that we can retrieve $\varphi_s = s/r$ for each $s \in [0, r - 1]$ without even knowing a single eigenvector of $U_x$, but there is a problem. Say that we want to find the order of $x = 3$ modulo $N = 11$, which is $r = 5$, and suppose that the true phase that the algorithm \tit{should} output is $$\varphi_s = \dfrac{s}{r} \implies \varphi_2 = \dfrac{2}{5} = 0.4$$ The QPE algorithm, however, cannot output this exact value, because 0.4 cannot be written precisely in binary --- in particular, the reason is that 5 does not divide $2^t$ for any $t$. This means that our algorithm will return an \tit{approximation} of $\varphi_2$. Say that we are working with 6 registers of precision and fix $t = 6$; then QPE will output the closest approximation of 0.4 with 6 bits: $$\tilde \varphi_2 = 0.40625 = \dfrac{13}{32}$$ Now, it would be a mistake to assume that $r = 32$! How do we proceed?

\begin{frameddefn}{Continued fraction}
    Given a rational number $x \in \Q$, the continued fraction of $x$ is composed by a set of numbers $a_0, \ldots, a_n \in \N$ --- where $a_n \neq 0$ --- such that $$x = a_0 + \dfrac{1}{a_1 + \dfrac{1}{a_2 + \dfrac{1}{\ddots + \dfrac{1}{a_n}}}}$$ The continued fraction is usually denoted with the following notation $$x = [a_0; a_1, a_2, \ldots , a_n]$$
\end{frameddefn}

Moreover, each \curlyquotes{partial} continued fraction
\begin{equation*}
    \begin{split}
        & [a_0] \\ 
        & [a_0; a_1] \\ 
        & [a_0; a_1, a_2] \\ 
        & \ldots \\ 
        & [a_0; a_1, a_1, \ldots, a_n] \\ 
    \end{split}
\end{equation*}
is called \tbf{convergent} of the original number.

For instance, the number $\tfrac{338}{121}$ can be written as follows: $$\dfrac{338}{121} = [2;1,3,1,5,4]$$ and its convergents are
\begin{equation*}
    \begin{split}
        & [2] \\ 
        & [2;1] \\
        & [2;1, 3] \\ 
        & \ldots \\ 
        & [2;1, 3, \ldots, 4] \\ 
    \end{split}
\end{equation*}
We will use continued fractions to recover $r$ from $\tilde \varphi_s$. Suppose that QPE outputs $\tilde \varphi_2 = \tfrac{13}{32}$ as in the previous example. Without going into the details, there is a classical algorithm which is able to recover its continued fraction in $O(L^3)$ time --- where $L = \ceil{\log N}$. Thus, we get that $$\tilde \varphi_2 = 0 + \dfrac{1}{2 + \dfrac{1}{2 + \dfrac{1}{6}}} = [0;2,2,6]$$ Now, to recover $\tfrac{2}{5}$ we just need to compute the convergents:
\begin{equation*}
    \begin{split}
        & [0] = 0 \\ 
        & [0;2] = \dfrac{1}{2} \\ 
        & [0;2,2] = \dfrac{2}{5} \\ 
        & [0;2, 2, 6] = \dfrac{13}{32} \\
    \end{split}
\end{equation*}
Finally, since we know that $r$ is the order of $x$ modulo $N$, we just need to pick the convergent with the smallest denominator $r$ such that $x^r \equiv 1 \bmod N$. In our example we had $x = 3$ and $N = 11$, thus
\begin{equation*}
    \begin{split}
        & 3^2 \equiv 9 \not\equiv1 \bmod 11 \\
        & 3^5 \equiv 243 \equiv 1 \bmod 11 \\ 
    \end{split}
\end{equation*}
This is how we can recover $r = 5$ --- to be precise, there are theoretical results which guarantee that $\tfrac{2}{5}$ shows among the convergents of $\tfrac{13}{32}$ but this is ouside the scope of our discussion.

\section{Shor's algorithm}

TODO \todo{intro}

\begin{framedthm}{Fundamental Theorem of Arithmetic}
    Given any $N \in \N$, there exists unique $p_1, \ldots, p_m \in \Primes$ primes and $\alpha_1, \ldots, \alpha_m$ exponents --- for some $m \in \N$ --- such that $$N = p_1^{\alpha_1} \cdot \ldots \cdot p_m^{\alpha_m}$$
\end{framedthm}

The set of primes $p_1, \ldots, p_m \in \Primes$ is called \tbf{prime factorization} of $N$. The \tbf{Integer Factorization Problem (IFP)} asks to retrieve the prime factorization of any given $N \in \N$.

\begin{framedthm}{}
    Given any $L$-bit long composite natural number $N \in \N$, let $x \in [2, N]$ different from $N - 1$ be a solution to the equation $$x^2 \equiv 1 \bmod N$$ Then, at least one of $\gcd(x- 1, N)$ and $\gcd(x + 1, N)$ is a non-trivial factor of $N$ that can be computed using $O(L^3)$ operations.
\end{framedthm}

\begin{framedthm}{}
    Given any $N \in \N$ odd composite positive integer with $m$ prime factors, let $x \in_R [1, N - 1]$ such that $\gcd(1, N - 1) = 1$. Then, if $r$ is the order of $x$ modulo $N$, it holds that $$\Pr[\mbox{$r$ is even and $x^{r/2} \not \equiv 1 \bmod N$}] \ge 1 - \dfrac{1}{2^m}$$
\end{framedthm}

TODO \todo{write the alg and explain stuff?}

TODO \todo{size of QPE}
TODO \todo{write QPE alg}

TODO \todo{size of Ux gates}
